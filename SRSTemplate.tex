%\documentclass[12pt]{IEEEtran}
\documentclass[12pt, a4paper]{article}
\usepackage{fancyhdr}

\begin{document}

\begin{titlepage}
	\begin{center}
	\huge{University of Pretoria\\
	Software Engineering 301 - COS 301}\\
	\line(1,0){400}\\
	\huge{\bfseries NavUp Software Requirements Specification}\\
	\line(1,0){200}\\
	Team Teal\\
	16 February 2017\\
	[3cm]
	\end{center}
	\begin{flushleft}
	\bfseries{Author(s):}
	\end{flushleft}
	\begin{flushleft}
	Tlaka Mankgwanyane	\hspace{10mm}{\textbf {u14351872}}\\
	Alberts Josef			\hspace{26mm}{\textbf{u14395283}}\\
	Oratile Motswagosele	\hspace{11mm}{\textbf{u15306195}}\\
	Carrim Muhammed		\hspace{14mm}{\textbf{u15019854}}\\
	Jackson Pearce			\hspace{22.5mm}{\textbf{u14044332}}\\
	Potgieter Linda		\hspace{21.6mm}{\textbf{u14070091}}\\
	Kanda Madimba			\hspace{19.5mm}{\textbf{u17289077}}\\
	\end{flushleft}
\end{titlepage}

\tableofcontents

\section{Introduction}
\subsection{Purpose} 
This is the system requirement specification (SRS) aims to refine and expand on the required capabilities of the NavUP system. This includes discussion of the individual low coupled subsystems, functional and quality requirements. This document is intended for the developers of the NavUP system as well as the clients who own the system. It will refine what is required of the system, what the main purpose of the system will be and what additional capabilities it will have. 
\subsection{Scope}
The system being designed will be able to guide a variety of users such as students, lecturers and visitors, through the University of Pretoria's (UP) various campuses. This system will be identified as the NavUP system, referencing the navigation it will provide to visitors of the UP campuses. The system will be able to route users between buildings and on a campus, as well as guiding them to the chosen lecture hall within the building. Push notifications will also be sent to the users through the application when he/she is near a venue where public events are currently or will in future occur. There will not be any push notifications sent if the users is not near the venue where these events will occur. The user will be able to create a profile which will unlock additional functionalities such as sharing locations with friends, saving frequently used places, and adding timetable integration. The application will also be able to notify a user about high traffic areas which the user may want to avoid in order to arrive earlier at his/her destination. The system aims to simplify all users' navigation through the various UP campuses, ensuring quicker navigation to destinations and avoidance of congested routes.     
\subsection{Definitions, Acronyms, and Abbreviation}
SRS - System Requirement Specification
UP - University of Pretoria
NavUp - The system being designed, acronym for Navigate(Nav) University of Pretoria (UP).
System - The NavUp system that is being designed.
Product/Application - NavUp system
Traffic - Areas in which there is a higher concentration of users which may affect arrival times.
Hot Spots - Areas in which there are wi-fi access for the users.
\subsection{Overview}
This document will provide more details abut the product (NavUp), including different interfaces, memory requirements, operations, as well as site adapt ion requirements. Functions, characteristics, constraints, and dependencies will be discussed. Lastly the document will elaborate on  the different requirements, including external interface and functional requirements.

\section{Overall Description}
	\subsection{Product Perspective}
		\subsubsection{System Interfaces}
		\subsubsection{User Interfaces}
		\subsubsection{Hardware Interfaces}
		\subsubsection{Software Interfaces}
		\subsubsection{Communication Interfaces}
		\subsubsection{Memory}
		\subsubsection{Operations}
		\subsubsection{Site Adaptation Requirements}
	\subsection{Product Functions}
	\subsection{User Characteristics}
	\subsection{Constraints}
	\subsection{Assumptions and Dependencies}

\section{Specific Requirements}
	\subsection{External Interface Requirements}
	\subsection{Functional Requirements}
	\subsection{Performance Requirements}
		\subsubsection{Position Accuracy} The position of a device with NavUP activated should be accurately determined by the system - with no more than 15m of deviation from actual location of the device.
		\subsubsection{Time to Determine Position} It should take NavUP no longer than 45 seconds to determine the location of the device (with reasonable accuracy) once the application has been opened.
		\subsubsection{Immidiacy of Push Notifications} Relevant push notifications should be pushed to the user's device no further than 30m from the focus of said notification. For example, current events at the AULA should not be displayed if a user is further than 30m from the AULA, to prevent cluttering of the notification bar.
		\subsubsection{View/Location Updates} Updates of the user's current location as displayed on the device screen should take place at intervals of no more than 5 seconds.
		\subsubsection{Application Size} The application, once installed, should consume no more than 150mb of storage space on the user's device. This includes cached maps and saved profile data.
		\subsubsection{User Login} Upon providing the system with correct credentials, the application must present the user with the navigation screen within 7 seconds. 
	\subsection{Design Constraints}
	\subsection{Software System Attributes}
\end{document}